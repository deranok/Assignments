%preamble stuff
\documentclass[12pt]{article}
\rmfamily % font
%packages
\usepackage{amsmath}
\usepackage{textcomp}
\usepackage{wasysym}
\usepackage[hmargin=3cm,vmargin=3.5cm]{geometry}
\usepackage{tikz}
%advanced settings
\setlength\parindent{0pt} %no indent for all paragraphs
%opening
\title{\textbf{UNIVERSITY OF GHANA BUSINESS SCHOOL\\ \vspace{0.5cm}
	First Semester, 2017/2018\\ % \vspace{0.5cm}
	UGBS 603: Economics\\
	Assignment 2}}

\date{Deadline: 10th October, 2017}
\author{10423631 -- Oppey Geoff}

\begin{document}
\maketitle
\clearpage

\section*{Solution to Problem 1}

a) The completed table is shown below.\\
\begin{center}
	\begin{tabular}{|c|c|c|c|c|c|c|c|}
		\hline
		Q	& TFC  & TVC  & TC   & MC   & AFC   & AVC   & ATC \\ 
		\hline
		0	& 15   & 0    & 15   & -    & -     & -     & -    \\ 
		1	& 15   & 3    & 18   & 3    & 15.0  & 3     & 18   \\ 
		2	& 15   & 5    & 20   & 2    & 7.5   & 2.5   & 10   \\ 
		3	& 15   & 6    & 21   & 1    & 5.0   & 2.0   & 7    \\ 
		4	& 15   & 8    & 23   & 2    & 3.75  & 2.0   & 5.75 \\ 
		5	& 15   & 13   & 28   & 5    & 3     & 2.6   & 5.6  \\
		6	& 15   & 22   & 37   & 9    & 2.5   & 3.67  & 6.17 \\
		7	& 15   & 36   & 51   & 14   & 2.14  & 5.14  & 7.29 \\
		8	& 15   & 56   & 71   & 20   & 1.88  & 7.0   & 8.88 \\
		\hline
	\end{tabular}
\end{center}
	\clearpage

\section*{Solution to Problem 2}
a) Given that there is free entry into the pizza market and the product are differentiated, there is a high possibility that the market is a monopolistic competitive market.

However, given 10 firms in the market as being `numerous', there is an implication that there are barriers to entry into this market. Hence the market is an oligopoly.\\

%\vspace{0.5cm}

b) In a perfect competitive market, there is free entry and exit. Also, no supplier or consumer can change the price. If prices are high and the suppliers are making supernormal profits, new entrants will be attracted to the market. The market would be flooded with excess supply, leading to falling prices. If prices are low and suppliers are making subnormal profits, they would leave the market, leading to shortages and prices will increase. Eventually, in the long run, prices will become stable at the break-even point. Hence firms neither make profits nor losses in the long run.\\

c) A firm in a monopolistic competitive market holds a monopoly over his own product. Also, he faces a downward sloping demand curve. To maximize revenue, he would charge the price that occurs at the point where marginal revenue is equal to marginal cost.  
In the short run, a firm in a monopolistic competitive market will most likely be one of the few suppliers in the market. So his profit maximizing price would be above the average total cost, allowing him to earn supernormal profits. This is shown in figure [write the short run figure here].
In the long run, other firms would join the market, causing demand to become less elastic as supply increases. There would be new entrants into the market until no firm is able to charge prices above the average cost. Hence in the long run, the monopolistic competitive firm will charge a price that is just about equal to the average cost. This is shown in figure [figure for the long run here].
\clearpage


\section*{Solution to Problem 3}

a) Calculation of profit.
\begin{center}
	\begin{tabular}{|c|c|c|c|}
		\hline
		Q  & TC & TR &  Profit = TR - TC \\
		\hline
		0  & 8  & 0  &  -8\\
		1  & 9  & 8  & -1\\
		2  & 10 & 16 & 6\\
		3  & 11 & 24 & 13\\
		4  & 13 & 32 & 19\\
		5  & 19 & 40 & 21\\
		6  & 27 & 48 & 21\\
		7  & 37 & 56 & 19\\
		\hline
	\end{tabular}
\end{center}

To maximize profits, the firm should produce 6 units.

b) Calculation of marginal revenue and marginal costs.
\begin{center}
	\begin{tabular}{|c|c|c|}
		\hline
		Q & $ MC = \triangle TC/\triangle Q $                & $ MR = \triangle TC/\triangle Q $ \\
		\hline
		0 & -                                                & - \\
		1 & $\left(9 - 8\right) / \left(1 - 0\right) = 1$    & $ \left(8 - 0\right) / \left(1 - 0\right) = 8$ \\
		2 & $\left(10 - 9\right) / \left(2 - 1\right) = 1 $  & $ \left(16 - 8\right) / \left(2 - 1\right) = 8 $ \\
		3 &$ \left(11 - 10\right) / \left(3 - 2\right) = 1 $ & $ \left(24 - 16\right) / \left(3 - 2\right) = 8 $ \\
		4 &$ \left(13 - 11\right) / \left(4 - 3\right) = 2$  & $ \left(32 - 24\right) / \left(4 - 3\right) = 8 $ \\
		5 &$ \left(19 - 13\right) / \left(5 - 4\right) = 6  $& $ \left(40 - 32\right) / \left(5 - 4\right) = 8 $ \\
		6 &$ \left(27 - 19\right) / \left(6 - 5\right) = 8 $ & $ \left(48 - 40\right) / \left(6 - 5\right) = 8 $ \\
		7 &$ \left(37 - 27\right) / \left(7 - 6\right) = 10 $& $ \left(56 - 48\right) / \left(7 - 6\right) = 8 $ \\
		\hline
	\end{tabular}\\
\end{center}
	\vspace{0.5cm}
	The graph is shown below:
	[Insert graph here]
	
	c) The firm makes a constant marginal revenue, regardless of the level of output. This implies that the firm is in a perfect competitive market. 
	The firm makes supernormal profits at the point where Marginal Revenue equals Marginal Cost. Implying that it is operating in the short run.


\clearpage	

\section*{Solution to Problem 4}
a) False. The law of diminishing marginal returns tells us that as more of a variable input is applied on a fixed input, output would initially increase then it would reach a maximum and eventually falls. It helps to explain the shape of the average cost curve in the short run. The law of diminishing marginal returns has nothing to do with the long run average cost curve.

Instead, the long run average cost curve's shape is determined by the economies of scale the firm enjoys. In the long run, because the firm acquires additional fixed inputs, all inputs are variable. The firm may gain some savings from expanding the scale of operations. For example, discounts for buying in bulk. Therefore, the long run average cost curve may start to fall. As the scale of operations keep increasing, the long run average cost curve will keep decreasing and will reach a minimum. When the scale of operations is increased beyond this minimum, the firm will find it harder to control costs. For example, regulations increase as the size of the firm increases. The long run average cost starts to increase. This explains the shape of the long run average cost.\\

b) False. On the contrary, by facing a downward sloping demand curve, the monopolist must charge lower prices in order to increase the quantity it can sell. Hence, it can set the price and accept the quantity that the buyers are willing to buy or it can supply the quantity it wants and accept the market price for the quantity supplied. The monopolist, therefore, can control price or quantity but not both.\\

c) True. Average total cost is higher than marginal cost when the average total cost begins to increase.

  The total average cost curve is the total cost divided by the quantity. It can be represented as $ATC = AFC + AVC$. As the output increases, AFC keeps getting closer to zero and the ATC gets closer to the AVC. On the other hand $ MC = \triangle TC / \triangle Q $. Since fixed inputs do not change with output, $ MC = \triangle VC / \triangle Q $. Therefore, the main difference between ATC and MC at initial levels of output is AFC. 
 
  From the above, it can be seen that at initial levels of output, where there are increasing marginal returns, marginal cost will be falling. However, at those initial levels of output, the average fixed cost is high, causing the average total cost to be higher than marginal revenue. As output increases, the average total cost keeps falling (since there are still increasing marginal returns). In other words, when both average cost and marginal cost are falling, average cost will be higher than marginal cost.
  
  When diminishing marginal returns sets in, both marginal cost and average total cost begin to rise. However, since average fixed cost gets closer to zero, the average total cost will be closer to average variable cost. However, the marginal cost still remains as the full change in variable cost, which keeps rising. Since the average variable cost is lower than the change in variable cost at those high levels of output, average total cost is lower than marginal cost. In other words, when marginal cost and average total cost are rising, marginal cost is higher than average total cost.\\
  
  d) True. This can be explained as follows. $ATC = AVC + AFC$ and $AVC = VC/Q$. Therefore, the only difference between Average Total Cost and Average Variable Cost is Average Fixed Cost. Fixed Cost is constant so Average Fixed Cost gets closer to zero as the output gets larger. This causes the gap between the Average Total Cost and Average Variable Cost to get smaller. In other words, the Average Total Cost gets closer to the Average Variable Cost as the output increases.\\
  
  e) False. The monopolist can make losses. This is because it faces a downward sloping demand curve. Therefore in order to make higher sales, it must decrease prices. When prices are too low, it would make losses. The table below illustrates the point.
  
  \begin{center}
  	\begin{tabular}{|c|c|c|c|c|c|c|}
  		\hline
  		Q  & Price & TR & VC & FC  & TC & Profit = TR - TC\\
  		\hline
  		0  &  12   &  0 & 0  &  3  &  3 &   -             \\
  		1  &  11   & 11 & 3  &  3  &  6 &   5             \\
  		2  &  10   & 20 & 6  &  3  &  9 &   11            \\
  		3  &  9    & 27 & 9  &  3  & 12 &   15            \\
  		4  &  8    & 32 & 12 &  3  & 15 &   17            \\
  		5  &  7    & 35 & 15 &  3  & 18 &   17            \\
  		6  &  6    & 36 & 18 &  3  & 21 &   15            \\
  		7  &  5    & 35 & 21 &  3  & 24 &   11            \\
  		8  &  4    & 32 & 24 &  3  & 27 &   5             \\
  		9  &  3    & 27 & 27 &  3  & 30 &   -3            \\
  		10 &  2    & 20 & 30 &  3  & 33 &   -13           \\
  		\hline
  	\end{tabular}
  \end{center}
  
  In order to sell higher quantities, lower prices are offered by the monopolist. At about 10 units of output, the monopolist begins to make losses because the price is too low. Therefore the monopolist can make losses.\\
  
  f) True. In the long, more firms would flood the market. This would cause prices to fall. This would continue until prices are so low that all the firms breaks even. The prices cannot get lower because the firms would have to take losses. 
 
 
 \section*{Solution to Problem 5}
 a) For a firm to have diminishing returns to labour means that the next worker hired would add less to the output compared to the previous worker, or the previous worker added more to production than the next worker hired.\\
 
 In order to find out whether the production function $Q = 2L^{\frac{1}{2}}K^{\frac{1}{2}}$ exhibits diminishing returns to labour, we hold K constant and double L to see its effect on output.
 
 K=1, L=1: $Q = 2 \times 1^{\frac{1}{2}} \times 1^{\frac{1}{2}} = 2 \times 1 \times 1 = 2 $,
 
 K=1, L=2: $Q = 2 \times 2^{\frac{1}{2}} \times 1^{\frac{1}{2}} = 2 \times 1.41 \times 1 = 2.82 $.
 
 Doubling labour led to less than double output. Therefore, this production function exhibits decreasing returns to labour.\\
 
 b) In a firm with increasing returns to scale, an increase in all factors of input lead to a more than proportional increase in output. 
 
  In order to find out whether the production function $Q = 2L^{\frac{1}{2}}K^{\frac{1}{2}}$ exhibits diminishing returns to scale, we double K and L to see whether output increases more than proportionately.
  
  K=1, L=1: $Q = 2 \times 1^{\frac{1}{2}} \times 1^{\frac{1}{2}} = 2 \times 1 \times 1 = 2 $,
  
  K=2, L=2: $Q = 2 \times 2^{\frac{1}{2}} \times 2^{\frac{1}{2}} = 2 \times 1.41 \times 1.41 = 4 $.
  
  Doubling labour and capital led to exactly double output. Therefore, this production function does not exhibit increasing returns to scale.\\
  
  c) 
\end{document}