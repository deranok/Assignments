%preamble stuff
\documentclass[12pt]{article}
\rmfamily % font
%packages
\usepackage{amsmath}
\usepackage{textcomp}
\usepackage{wasysym}
\usepackage[hmargin=3cm,vmargin=3.5cm]{geometry}
\usepackage{tikz}
\usepackage{graphicx}
%advanced settings
\setlength\parindent{0pt} %no indent for all paragraphs
%opening
\title{\textbf{UNIVERSITY OF GHANA BUSINESS SCHOOL\\ \vspace{0.5cm}
	First Semester, 2017/2018\\ % \vspace{0.5cm}
	UGBS 603: Economics\\
	Assignment 3}}

\date{Deadline: 21st December, 2017}
\author{10423631 -- Oppey Geoff}

\begin{document}
\maketitle
\thispagestyle{empty} %remove the page number
\clearpage

\section*{Solution to Problem 1}
\setcounter{page}{1} %set the page counter to 1
a) The nominal GDP is calculated by multiplying the units produced by the current price. It is shown in the table below:

\begin{tabular}{|l|r|r|}
	\hline
	Product    & Current Year (2013)         & Base Year (2012)         \\
	\hline
	Apples     & 4,000 $\times$ 3 = 12,000   & 3000 $\times$ 2 =  6,000 \\
	Bananas    & 14,000 $\times$ 2 = 28,000  & 6000 $\times$ 3 = 18,000 \\
	Pineapples & 7,500 $\times$ 2 = 15,000   & 4000 $\times$ 1.5 = 6,000\\
	Oranges    & 32,000 $\times$ 5 = 160,000 & 8000 $\times$ 4 = 32,000 \\
	\hline
	Total      & 215,000                     & 62,000 \\
	\hline
\end{tabular}

\begin{equation*}
	\begin{split}
		\text{Percentage Increment}  &= \frac{\text{Current Nominal GDP - Base Nominal GDP}}{\text{Base Nominal GDP}} \times 100\\
		&= \frac{215,000 - 62,000}{62,000} \times 100\\
		&= \frac{153,000}{62000} \times 100\\
		&= 2.47 \times 100\\
		&= 247 \%
	\end{split}
\end{equation*}\\
\vspace{0.5cm}

b) The real GDP is calculated by multiplying the current units produced by the base year's price. It is shown in the table below:

\begin{tabular}{|l|r|r|}
	\hline
	Product    & Current Year (2013)         & Base Year (2012)         \\
	\hline
	Apples     & 4,000 $\times$ 2 = 8,000   & 3000 $\times$ 2 =  6,000 \\
	Bananas    & 14,000 $\times$ 3 = 42,000  & 6000 $\times$ 3 = 18,000 \\
	Pineapples & 7,500 $\times$ 1.5 = 11,250   & 4000 $\times$ 1.5 = 6,000\\
	Oranges    & 32,000 $\times$ 4 = 128,000 & 8000 $\times$ 4 = 32,000 \\
	\hline
	Total      & 189,250                     & 62,000 \\
	\hline
\end{tabular}

\begin{equation*}
\begin{split}
\text{Percentage Increment}  &= \frac{\text{Current Real GDP - Base Real GDP}}{\text{Base Real GDP}} \times 100\\
&= \frac{189250 - 62000}{62000} \times 100\\
&= \frac{127250}{62000} \times 100\\
&= 2.05 \times 100\\
&= 205 \%
\end{split}
\end{equation*}\\
\vspace{0.5cm}

c) \begin{equation*}
	\text{GDP Deflator} = \frac{\text{Nominal GDP}}{\text{Real GDP}} \times 100
\end{equation*}\\
\vspace{0.3cm}
For current year:
\begin{equation*}
	\begin{split}
		\text{GDP Deflator} &= \frac{215000}{189250} \times 100\\
		&= 1.14 \times 100\\
		&= 114\\
	\end{split}
\end{equation*}

For base year:
\begin{equation*}
\begin{split}
\text{GDP Deflator} &= \frac{62000}{62000} \times 100\\
&= 1.00 \times 100\\
&= 100\\
\end{split}
\end{equation*}

The percentage change in price level is 14\% (114 - 100).\\
\vspace{0.5cm}

d) The change in Nominal GDP (247\%) can be attributed to a change in physical volume, which is measured by the real GDP of (205\%) and a change in prices of (42\%). Therefore, in my opinion, the percentage increase in Nominal GDP is due more to increase in volume than increase in prices.

\section*{Solution to Problem 2}
a) Contribution to GDP is GHS 30 (GHS 15 - GHS 12 $\times$ 10). The reason being that previous year's expenses are not considered in calculating the GDP.
\\

b) Contribution to GDP is GHS 120,000 (GHS 2m $\times$ 6\%). The brokerage service was offered during the year but the gun was not produced in the year.
\\

c) Contribution to GDP is  GHS 56,000 (GHS 16,000 + GHS 40,000). Given that the nanny's salary can be captured by the statistician.
\\

d) Contribution to GDP is zero. Lottery winnings do not contribute to GDP.
\\

e) Contribution to GDP is GHS 5,000. Appearance in an advert contributed to GDP.
\\

f) Contribution to GDP is GHS 120m (GHS 100m + GHS 20). Being GHS 100m income on sale of new computers and GHS 20m profit on sale of old computers.
\\
\vspace{0.5cm}

\section*{Solution to Problem 3}
a) The number of people employed includes those who answer yes to Q1, Q2 and Q3.
\begin{align*}
	Q1 + Q2 + Q3 &= 29861 + 496 + 2394\\
	&= 32751\\
\end{align*}

b) The number of people that are unemployed includes those who answer yes to Q5. That is, 419.\\

c) \begin{align*}
	\text{Labour Force} &= \text{Employed} + \text{Unemployed}\\
	&= 32751 + 419\\
	&= 33170\\
\end{align*}\\

d) \begin{align*}
	\text{Labour Force Participation Rate} &= \frac{\text{Labour Force}}{\text{Population (above fifteen)}} \times 100\\
	&= \frac{33170}{42697} \times 100\\
	&= 0.78 \times 100\\
	&= 78\%\\
	\end{align*}\\
	
e) \begin{align*}
	\text{Unemployment rate} &= \frac{\text{Employed}}{\text{Labour Force}} \times 100\\
	&= \frac{419}{33170} \times 100\\ 
	&= 0.0126 \times 100\\
	&= 1.26\%\\
   \end{align*}

\section*{Solution to Problem 4}
a) Deflation is the persistent fall in the general price level over a period of time in an economy. The formular is given as follows, given that P is the general price level.
\begin{equation*}
	\pi_{t+1} = \frac{P_{t+1} - P_{t}}{P_{t}} \times 100\\
\end{equation*}
\\

b) Disinflation refers to a period of declining inflation rate whiles deflation refers to a general fall in prices in the economy. In other words, while disinflation refers to prices increasing at a decreasing rate, deflation refers to general prices falling and inflation less than 0\%.
\\

c) The inflation for the month of August may be computed as follows.
\begin{equation*}
	\begin{split}
		\pi_{\text{August}} &= \frac{CPI_{2017} - CPI_{2016}}{CPI_{2016}} \times 100\\
		&= \frac{201.3 - 179.2}{179.2} \times 100\\
		&= \frac{22.1}{179.2} \times 100\\
		&= 0.123 \times 100\\
		&= 12.3\%\\
	\end{split}
\end{equation*}\\

d) Since January 2012 is the base month, the percentage change in the cost of living can be computed as follows.

\begin{equation*}
	\begin{split}
	\%\triangle \text{Cost of living (August 2017)} &= 201.3 - 100\\
	&= 101.3\%\\
	\end{split}
\end{equation*}\\

e) Inflation leads to shoe leather and menu costs. Menu costs are the costs associated with re-pricing items sold in order to bring them in line with the general inflation rate. Shoe leather costs are the costs associated with frequently visiting the bank in order to make withdrawals. Since if inflation is high, less money is held for transaction but is left to earn interest in the bank.

Inflation can cause a fall in investments. Firms may anticipate that interest rates will have to rise to deal with inflation, and this will reduce their confidence, leading to a fall in their ability to make capital investments.\\

f) On the other hand some benefits of inflation include.

Firstly, debtors generally benefit from inflation since the real amount of the debt keeps falling as inflation rises.

Also, a moderate amount of inflation in the economy provides an incentive for firms to keep producing. If price levels are falling, firms would have to cut back on production in order to minimize losses.

\section*{Solution to Problem 5}
a) Monetary policy refers to the actions of the central bank of a country that seeks to control economic activities through the regulation of money supply.\\

b) Loose, or expansionary monetary policy seeks to boost economic activities through an increase in the supply of money. This can be achieved by reducing the discount rate or reserve requirements provides banks with an incentive to loan money and make credit available. Tight, or contractionary monetary policy seeks to slow down economic activities through a decrease in the supply of money. This can be achieved by increasing the discount rate or reserve requirements provides banks.\\

c) The monetary policy rate is the rate at which the central bank lends to other banks within the economy. A reduction in the monetary policy rate makes it cheaper to borrow from the central bank, which serves as an incentive to provide more loans to the general public and increase the money supply. Therefore, this is an expansionary monetary policy.\\

d) In the money markets, the increase in money supply without a corresponding decrease in interest rates will cause the demand for money to be less than the new supply of money. Therefore, in order to establish a new equilibrium, a lower interest rate will be set in order to increase the demand for money within the market. This is shown in the diagram below.

\begin{center}
	\includegraphics[width=10cm,height=10cm]{money_supply.jpg}
\end{center}

At the initial equilibrium A, interest rates were $i^{'}_{\$}$ and money supply was at point 1. Through a monetary policy rate reduction, money supply will increase from $M^{S^{'}}/P_{\$}$ to $M^{S^{''}}/P_{\$}$ due to banks being able to borrow at a cheaper rate from the central bank. Iff interest rates remain at $i^{'}_{\$}$, the demand for money will remain at point 1, which is lower than than the supply of money at point 2. Therefore, in order to establish a new equilibrium, interest rates will need to fall to $i^{''}_{\$}$, where the demand for money equals the supply of money.\\

e) Since interest rates are falling, the opportunity cost of holding physical cash instead of investment will decline, hence less people will be willing to hold interest bearing assets such as bonds. This will lead to a fall in the investment expenditure. Another reason for this is that low interest rates will cause the market value of interest bearing assets to rise. Hence less people will be willing to buy interest bearing assets, leading to a fall in investment expenditure.

On the other hand, consumption expenditure will rise since people will be willing to hold more money for transactions. Also, lower interest rates will lead to a rise in the value of interest bearing assets. Hence more people will want to sell off their investments and cash in. This would make more money available for consumption, hence consumption expenditure will rise.
\end{document}